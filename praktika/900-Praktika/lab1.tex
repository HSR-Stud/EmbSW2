\subsection{Aufgabe 1: Realisierung einer einfach verketteten Liste}
Realisierung einer einfach verketteten Liste
Implementieren Sie eine einfach verkettete Liste für double-Werte in C++ mit der Entwicklungsumgebung
Eclipse. Die Listenklasse erhält den Namen SList.
SList soll die nachfolgenden Methoden anbieten, wobei die geeigneten Parameter noch zu definieren sind.
Achten Sie auf den korrekten Einsatz von const.
\begin{itemize}
\item Nehmen Sie als Ausgangspunkt das vorgegebene Eclipse-Projekt. Ergänzen Sie bei Bedarf auch das einfache
Testprogramm.
\item Konstruktor (Ctor) Initialisiert eine leere Liste.
\item Destruktor (Dtor) Löscht die gesamte Liste und gibt den allozierten Speicher wieder frei.
\item \texttt{insertAt} Fügt ein Element an (nach) der bezeichneten Position ein. Einfügen am Anfang der Liste kann mittels Position 0 bewerkstelligt werden.
\item\texttt{deleteAt}
Löscht das Element an der bezeichneten Position.
\item\texttt{search}
Sucht das erste Element in der Liste mit dem definierten Wert und gibt die Position zurück. Falls das Element
nicht in der Liste vorhanden ist, wird als Position der Wert 0 zurückgegeben.
\item\texttt{isEmpty}
Gibt true zurück, falls die Liste leer ist.
\item\texttt{getNumber}
Gibt die Anzahl der Elemente der Liste zurück.
\item\texttt{getValue}
Gibt den Wert des Elementes an der bezeichneten Position zurück.
\item\texttt{setValue}
Setzt den Wert des Elementes an der bezeichneten Position.
\item\texttt{print}
Schreibt den Inhalt der Liste in die Console.
\end{itemize}

\subsubsection{Lösung}

\lstinputlisting[language=C++, style=C++, multicols=2]{900-Praktika/prak01/Loesung/Lists/src/SList.h}
\noindent\makebox[\linewidth]{\rule{\paperwidth}{0.4pt}}
\lstinputlisting[language=C++, style=C++, multicols=2]{900-Praktika/prak01/Loesung/Lists/src/SList.cpp}
\noindent\makebox[\linewidth]{\rule{\paperwidth}{0.4pt}}
\lstinputlisting[language=C++, style=C++, multicols=2]{900-Praktika/prak01/Loesung/Lists/src/Lists.cpp}

\subsection{Aufgabe 2: Liste für beliebige Typen}

Erweitern Sie die Klasse SList so, dass beliebige Typen abgespeichert werden können. Achten Sie dabei
auch auf die korrekte Verwendung von Referenzen und des \texttt{const} Qualifiers.

Zur Erinnerung: Sie müssen hier Templates verwenden.

\subsubsection{Lösung}

\lstinputlisting[language=C++, style=C++, multicols=2]{900-Praktika/prak01/Loesung/A2/src/SList.h}
\noindent\makebox[\linewidth]{\rule{\paperwidth}{0.4pt}}
\lstinputlisting[language=C++, style=C++, multicols=2]{900-Praktika/prak01/Loesung/A2/src/ListTest.cpp}

\subsection{Aufgabe 3: Realisierung einer doppelt verketteten Liste}

Implementieren Sie die Klasse \texttt{DList}, welche eine doppelt verkettete Liste für double-Werte in C++ definiert.
\texttt{DList} soll dieselben Methoden anbieten wie bereits \texttt{SList} in Aufgabe 1.
Bemerkung: hier müssen Sie keine Templates verwenden.

\subsubsection{Lösung}

\lstinputlisting[language=C++, style=C++, multicols=2]{900-Praktika/prak01/Loesung/Lists/src/DList.h}
\noindent\makebox[\linewidth]{\rule{\paperwidth}{0.4pt}}
\lstinputlisting[language=C++, style=C++, multicols=2]{900-Praktika/prak01/Loesung/Lists/src/DList.cpp}

\subsection{Aufgabe 4: Gemeinsame Basisklasse}

Die beiden Listen \texttt{SList} und \texttt{DList} haben verschiedene Gemeinsamkeiten, eine gemeinsame Basisklasse
List bietet sich offensichtlich an. Implementieren Sie eine gemeinsame Basisklasse \texttt{List}. Achten Sie darauf,
dass die Sichtbarkeiten (\texttt{public}, \texttt{private}, \texttt{protected}) so gewählt werden, dass die Unterklassen Zugriff
auf die von ihnen benötigten Daten haben. Es ist jedoch keine Lösung, einfach alles als \texttt{public} zu deklarieren.

\subsubsection{Lösung}

\lstinputlisting[language=C++, style=C++, multicols=2]{900-Praktika/prak01/Loesung/A4/src/DList.h}
\noindent\makebox[\linewidth]{\rule{\paperwidth}{0.4pt}}
\lstinputlisting[language=C++, style=C++, multicols=2]{900-Praktika/prak01/Loesung/A4/src/DList.cpp}
\noindent\makebox[\linewidth]{\rule{\paperwidth}{0.4pt}}
\lstinputlisting[language=C++, style=C++, multicols=2]{900-Praktika/prak01/Loesung/A4/src/SList.h}
\noindent\makebox[\linewidth]{\rule{\paperwidth}{0.4pt}}
\lstinputlisting[language=C++, style=C++, multicols=2]{900-Praktika/prak01/Loesung/A4/src/SList.cpp}
\noindent\makebox[\linewidth]{\rule{\paperwidth}{0.4pt}}
\lstinputlisting[language=C++, style=C++, multicols=2]{900-Praktika/prak01/Loesung/A4/src/Lists.cpp}
\noindent\makebox[\linewidth]{\rule{\paperwidth}{0.4pt}}
\lstinputlisting[language=C++, style=C++, multicols=2]{900-Praktika/prak01/Loesung/A4/src/List.h}

\subsection{Aufgabe 5: Gemeinsame Basisklasse mit Templates}

Die komfortabelste Lösung ist offensichtlich, die beiden Klassen \texttt{SList} und \texttt{DList} als Templateklassen zu implementieren mit einer gemeinsamen Basisklasse.

\subsubsection{Lösung}

\lstinputlisting[language=C++, style=C++, multicols=2]{900-Praktika/prak01/Loesung/A5/src/DList.h}
\noindent\makebox[\linewidth]{\rule{\paperwidth}{0.4pt}}
\lstinputlisting[language=C++, style=C++, multicols=2]{900-Praktika/prak01/Loesung/A5/src/SList.h}
\noindent\makebox[\linewidth]{\rule{\paperwidth}{0.4pt}}
\lstinputlisting[language=C++, style=C++, multicols=2]{900-Praktika/prak01/Loesung/A5/src/List.h}
\noindent\makebox[\linewidth]{\rule{\paperwidth}{0.4pt}}
\lstinputlisting[language=C++, style=C++, multicols=2]{900-Praktika/prak01/Loesung/A5/src/ListTest.cpp}
