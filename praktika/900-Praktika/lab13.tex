\subsection{Aufgabe 1: Hashing}

Im Verzeichnis ./Vorgabe/Hash finden Sie drei JPG-Files, die alle dieselbe Grösse und denselben Timestamp
haben. Zwei Dateien sind identisch, eine ist unterschiedlich. Sie sollen herausfinden, welche Datei
unterschiedlich ist.
Sie könnten diese Aufgabe mit dem Linux-Befehl diff lösen. Sie könnten die Dateien ebenfalls Byte für
Byte binär vergleichen. Sie sollen nun aber diese Aufgabe mit Hilfe einer Hashfunktion lösen. Wenn die Dateien
identisch sind, haben Sie denselben Fingerprint.

\subsubsection{Lösung}
Im Windows Explorer kann unter dem Kontextmenu CRC SHA ein Hash Fingerprint einer Datei erstellt werden,
z.B. SHA-256. Unter Linux besteht für SHA-256 der Befehl sha256sum.
Die SHA-256 Hashes der einzelnen Dateien sind:

img3.jpg: SHA256: EF5CDDD830BBCBF544FFEB18E3B28B4D19D65A5AE16F4618D742683940F9E64D

img4.jpg: SHA256: EF5CDDD830BBCBF544FFEB18E3B28B4D19D65A5AE16F4618D742683940F9E64D

img5.jpg: SHA256: 5E8EDB576C3CF31013ED0858E35C280BA5B667164DA2991C1AE2ECAF90975A2B

Aus den erhaltenen Hashes sieht man, dass die Dateien img3.jpg und img4.jpg identisch sind, img5.jpg aber
abweicht. In img5.jpg wurde nur ein einziges Byte abgeändert, die Metainformation "Photoshop" wurde in
"Photishop" abgeändert, das eigentliche Bild ist identisch. Man sieht eindrücklich, dass die Änderung eines
einzigen Bytes einen völlig anderen Hash ergibt. Das ist unter anderem die Stärke einer guten Hashfunktion.

\subsection{Aufgabe 1: Verschlüsselten Text entschlüsseln}

Eine typische Verschlüsselungsmethode wandelt die Buchstaben eines Textfiles in ASCII und führt Buchstabe
um Buchstabe eine XOR-Verknüpfung mit einem Wert aus einem geheimen Schlüssel durch. Die XORFunktion
ist sehr schnell und sie ist symmetrisch, d.h.

\lstinputlisting[language=C++, style=C++, multicols=2]{900-Praktika/prak13/Loesung/Cipher/CipherFreq/src/CipherFreq.cpp}

\noindent
\texttt{
ciphertextByte = plaintextByte $\oplus$ key und \\
plaintextByte = ciphertextByte $\oplus$ key}

Je länger ein Passwort ist und je mehr unterschiedliche Zeichen es enthält, desto sicherer ist das Passwort.
In diesem Beispiel nehmen wir ein Passwort, das nur aus 3 kleinen Buchstaben (a-z) besteht. Der Schlüssel
besteht aus der fortlaufenden Aneinanderreihung dieses Passworts.
In ./Vorgabe/Cipher finden Sie einen ciphertext, der mit einem Passwort bestehend aus 3 Kleinbuchstaben
verschlüsselt wurde. Ihre Aufgabe besteht darin, das Passwort zu knacken und den Text zu entschlüsseln.
Zur Lösung können Sie die folgenden Kenntnisse nutzen:
\begin{itemize}
  \item Das Passwort besteht aus drei Kleinbuchstaben, der Schlüssel wird durch fortlaufendes Aneinanderreihen
  des Passworts generiert
  \item  Die Verschlüsselung geschieht mittels XOR-Funktion
  \item  Der unverschlüsselte englische Text besteht nur aus Gross- und Kleinbuchstaben, Ziffern, Leerzeichen
  und Satzzeichen
\end{itemize}

Als Angriffsvariante können Sie beispielsweise eine oder mehrere der folgenden Strategien anwenden:
\begin{itemize}
  \item Trial and error
  \item  Brute force (es gibt 26 * 26 * 26 = 17'576 Möglichkeiten für den Schlüssel)
  \item  Intuition, Raten, etc.
  \item  Häufigkeitsanalyse der Buchstaben. Die 10 häufigsten Buchstaben der englischen Sprache in abnehmender
  Reihenfolge sind: e, t, a, o, i, n, s, h, r, d, l.
  \item  Erraten einer Teillösung und fortlaufendes Erraten von zusätzlichen Buchstaben
  \item  Weitere intelligente Methoden (beachten Sie, dass der Plaintext aus lesbaren Zeichen besteht)
\end{itemize}
