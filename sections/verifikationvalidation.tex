\section{Verifikation und Validation}
\subsection{Softwareentwicklung}
Projektphasen:
\begin{itemize}
\item Analyse
	\begin{itemize}
		\item T"atigkeit: Analyse des Problems, Erarbeiten der Spezifikation
		\item Erzeugnis: Spezifikation der funktionalen und nicht funktionalen Anforderungen 
	\end{itemize}
\item Design 
		\begin{itemize}
			\item T"atigkeit: Architektur der Software festlegen (Grobdesign), Design der Units (Detaildesign)
			\item Erzeugnis Grobdesign: Beschreibung der Architektur, Festlegung der Schnittstellen der Units
			\item Erzeugnis Detaildesign: Beschreibung der Algorithmen (Aktivit"atsdiagramme), Beschreibung der Units (Klassendiagramme) 
		\end{itemize}
\item Implementation 
	\begin{itemize}
		\item T"atigkeit: Codierung des Designs, Durchf"uhren der Unittests
		\item Erzeugnis: Dokumentierter/Kommentierter Code, Testvorschriften und Protokolle der Unittests, Testprogramme  
	\end{itemize}
\item Test (dreistufig)
	\begin{itemize}
		\item Unittest
			\begin{itemize}
				\item Teil der Implementationsphase
				\item Whitebox-Test
				\item Durch Programmierer
			\end{itemize}
		\item Integrationstest
				\begin{itemize}
						\item Zusammenf"ugen mehrerer Units
						\item Blackbox-Test
						\item Schwerpunkt: Unit-Schnittstellen
						\item Nicht durch Programmierer
				\end{itemize}
		\item Systemtest
				\begin{itemize}
						\item Testen des Gesamtsystems in Originalumgebung
						\item Blackbox-Test
						\item Schwerpunkt: Externes Verhalten des Systems
						\item Nicht durch Programmierer
				\end{itemize}
		\item Erzeugnis: Spezifikation der funktionalen und nicht funktionalen Anforderungen 
	\end{itemize}
\item Abnahmetest
	\begin{itemize}
		\item Ziel der oberen Tests: Fehler finden
		\item Ziel der Abnahmetest: Erf"ullung von festgelegten Kriterien f"ur geforderte Funktionen nachweisen 
		\item Durch Auftraggeber
	\end{itemize}
\end{itemize}


\subsection{Verifikation und Validation}
\textbf{Verifikation:}\\
Machen wir das Produkt richtig? Software soll den Anforderungen entsprechen.\\
\textbf{Validation:}\\
Machen wir das richtige Produkt? Die Software soll machen, was der Benutzer wirklich ben"otigt.\\
V\&V muss in jeder Phase angewendet werden.\\
\textbf{Ziele:}\\
Defekte entdecken und heraus finden, ob System brauchbar ist.\\
\textbf{Unterschied V\&V und Debugging:}\\
V\&V: Fehler finden\\ 
Debugging: Fehler lokalisieren und korrigieren\\
Fehlerterminologie:\\
\begin{itemize}
\item Person macht Fehler: \textbf{mistake}
\item Software hat Defekt: \textbf{defect,fault}
\item Wir Defekt gefunden: \textbf{finding}
\item Ausf"uhren von defekter Software f"uhrt zum: \textbf{error}
\item Fehler kann zum Ausfall f"uhren: \textbf{failure}
\item Fehlerursache finden und flicken: \textbf{debugging}
\end{itemize}
Pr"ufverfahren:\\
\begin{itemize}
\item Review
\item Statische Tests (Ohne Ausf"uhrung des Programms)
\item Dynamische Tests (Durch Ausf"uhrung des Programms)
\item Simulation
\item Prototypen (Kritische Teile vorab untersuchen)
\end{itemize}

Anweisung"uberdeckung (instruction coverage):\\
Jeder Knoten (=Anweisung) muss mind. einmal besucht werden. 
Zweig"uberdeckung (branch coverage):\\
Jeder Zweig (=Verzweigung) muss mind. einmal besucht werden. 
Pfad"uberdeckung (path coverage):\\
Alle Pfade, d.h. alle Zweigkombinationen m"ussen besucht werden. 

\subsection{Testdurchf"uhrung}
Siehe Folien\\
Infinite Faults:\\
Wenn trotz Testen und Debugging die Fehlerzahl um einen unakzeptabel hohen Wert oszilliert.\\
Kosten f"ur zus"atzlich eliminierten Fehler: $\infty$\\
Massnahme: Projektabbruch

\subsection{Testen von EmbSys}
EmbSys in hardware-unabh"angige und hardware-abh"angige Teile strukturieren:\\
HW-unabh"angig: Unittests auf Entwicklungsplattform (PC)
HW-abh"angig: Externe Testhardware f"ur Driver und Stubs

\subsection{Reviewtechnik}
"Offentliche, verbale Begutachtungen eines erstellten Arbeitspakets. Zwischenmenschliche Angelegenheit, Erfolg h"angt von der Zusammenarbeit ab.
Warum ist Review gut? 
\begin{itemize}
\item Fehlerursachen finden nicht nur Symptome
\item Mit Reviews werden am meisten Fehler gefunden
\item Alle lernen was
\end{itemize}