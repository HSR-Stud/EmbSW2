\section{Linux Development Tools}
\subsection{Entwicklungstools}
\begin{description}
    \item [Integrated Development Environment (IDE) $\rightarrow$ Eclipse] \hfill \\
    Editor, Compiler, Linker, Debugger, etc. \\
    Viele Informationen über Code (-Wall)
    \item [Cross-Plattform und Proprietäre (herstellerspezifische) Software] \hfill \\
    Bsp. Qt (Cross-Plattform)
    \item [OSS]\\
    Die OSS-Community bietet laufend neue Tools, am meisten wird für Linux angeboten
\end{description}

\subsection{Häufige Fragen}
\begin{itemize}
  \item Welches Tool für Sourcecode-Debugging (evtl. multi-threaded)
  \item Speicherfehler ermitteln
  \begin{itemize}
    \item Memory leaks
    \item Buffer overflows
  \end{itemize}
  \item Codeüberdeckung (Code Coverage)
  \item Performance Analyse (Profiling)
\end{itemize}

\subsection{OSS-Tools für Linux}
\begin{itemize}
  \item Wir beschränken uns auf OSS-Tools für Linux
  \item GNU-Binutils
  \begin{itemize}
    \item GNU Binary Utilities (kurz binutils) sind eine Sammlung von Tools für Software-Entwickler
    \item Erzeugung und Manipulation von Programmen
    \item Analyse und Informationsbeschaffung aus Objektcode, Bibliotheken, Maschinencode etc.
    \item \url{http://www.gnu.org/software/binutils/}
  \end{itemize}
\end{itemize}

\subsection{GNU Binutils}
\begin{itemize}
    \item GNU Binary Utils $\rightarrow$ Sammlung von Tools für Software Entwickler (Open Source)
    \item Analyse Objektcode, Bibliotheken, Maschinencode
    \item Erzeugung / Manipulation von Programmen
    \item Um Tools nutzen zu können mit -g (Debugging Flags) kompilieren
\end{itemize}

\subsubsection{nm}
\begin{itemize}
  \item Lists symbols from object files
\end{itemize}
\begin{lstlisting}[style=bash]
$ nm CoffeMachine
...
00000000004164b4 T main
                 U malloc@@GLIBC_2.2.5
                 U memcpy@@GLIBC_2.14
                 U memset@@GLIBC_2.2.5
000000000040399a T _ZN17CoffeeMachineCtrl10resetWaterEv
00000000004043ae T _ZN17CoffeeMachineCtrl11resetCoffeeEv
...
$
\end{lstlisting}
Diese Namen sind mangled names.

\subsubsection{c++filt}
\begin{itemize}
    \item Filter to demangle encoded C++ symbols
\end{itemize}
\begin{lstlisting}[style=bash]
$ c++filt _ZN17CoffeeMachineTest12testBeansFSMEv
CoffeeMachineTest::testBeansFSM()
$
\end{lstlisting}

\subsubsection{gprof}
\begin{itemize}
    \item Display profiling information
    \item Ausführungszeiten von Programmen (Kann in Eclipse integriert werden)
    \item pollt Programm Counter mit 100Hz $\rightarrow$ sehr grobe Schätzung der Zeit
    \item Besser: valgrind (genauer)
\end{itemize}

\subsubsection{objdump}
\begin{itemize}
    \item Displays information from object files (many options)
    \item Z.B. -d $\rightarrow$ Disassembly erstellen
    \item Z.B. -S $\rightarrow$ Disassembly erstellen mit Sourcecode
\end{itemize}
\begin{lstlisting}[style=bash]
$ objdump -d CoffeeMachine
CoffeeMachine:    file format elf64-x86-64
Disassembly of section .init:
0000000000402978 <_init>:
  402978:    48 83 ec 08             sub   $0x8,%rsp
  40297c:    48 8b 05 65 a6 21 00    mov   0x21a665(%rip),%rax     # 61cfe8 <_DYNAMIC+0x200>
  402983:    48 85 c0                test  %rax,%rax
  402986:    74 05                   je    40298d <_init+0x15>
  402988:    e8 b3 00 00 00          callq 402a40 <__gmon_start__@plt>
  40298d:    48 83 c4 08             add   $0x8,%rsp
...
$
\end{lstlisting}
\begin{lstlisting}[style=bash]
$ objdump -S CoffeeMachine
...
  40382f:    be e0 2e 40 00          mov   $0x402ee0,%esi
  403834:    48 89 c7                mov   %rax,%rdi
  403837:    e8 04 f6 ff ff          callq 402e40 <_ZNSolsEPFRSoS_E@plt>
      if (evNoWater == e)
  40383c:    83 7d f4 04             cmpl  $0x4,-0xc(%rbp)
  403840:    75 5d                   jne   40389f <_ZN17CoffeeMachineCtrl12processWaterENS_5EventE+0xcb>
  403842:    48 8b 05 97 c8 21 00    mov   0x21c897(%rip),%rax
...
$
\end{lstlisting}

\subsubsection{size}
\begin{itemize}
    \item Lists the section sizes of an object file
\end{itemize}
\begin{lstlisting}[style=bash]
$ size CoffeeMachine
  text    data     bss      dec     hex filename
117517   11456   25768   154741   25c75 CoffeeMachine
$
\end{lstlisting}

\begin{description}
  \item[text:] code, const data $\rightarrow$ goes into ROM
  \item[data:] statically-allocated variables that are explicitly initialized to any value $\rightarrow$ goes into RAM
  \item[bss:] statically-allocated variables that are not explicitly initialized to any value $\rightarrow$ goes into RAM
  \item[dec:] text + data + bss
  \item[hex:] dec in hex
\end{description}

\subsection{Codeüberdeckung oder -abdeckung (code coverage)}
Bei der Ausführung eines Programms wird versucht, möglichst den ganzen
Code mindestens einmal auszuführen.
\begin{itemize}
    \item Anweisungsüberdeckung (instruction coverage): Jede Anweisung des Programms wird mindestens einmal ausgeführt
    \item Zweigüberdeckung (branch coverage): Jeder Programmzweig wird mindestens einmal durchlaufen
    \item Pfadüberdeckung (path coverage): Jeder Programmpfad, d.h. alle Zweigkombinationen, wird mindestens einmal durchlaufen
\end{itemize}
In der Praxis wird meist nur Anweisungsüberdeckung verlangt,
Zweigüberdeckung wird angestrebt (bei Schleifen explodiert die Anzahl Pfade).\\\\
Überdeckungsgrad \textless 100 \% bedeutet, dass mit den vorhandenen Testfällen nicht alle Anweisungen/Zweige/Pfade ausgeführt werden (ist u.a. auch ein gutes Gütemass für die Testfälle). Folgende Massnahmen können in diesem Fall vollzogen werden:
\begin{itemize}
    \item Zusätzliche Testfälle definieren
    \item Nicht überdeckter Code analysieren, z.B. mittels Code Inspection
\end{itemize}
Beim nicht ausgeführten Code kann es sich auch um toten Code handeln. Wenn logisch falsch $\rightarrow$ korrigieren bzw. entfernen. Wenn aufgrund defensiven Programmierstil (z.B. default: break) $\rightarrow$ lassen.
\subsubsection{Code Coverage Tool: gcov}
\begin{itemize}
    \item gcov kann in der Kommandozeile oder in Eclipse genutzt werden
    \item gcov zeichnet auf, wie oft eine Codezeile ausgeführt wird
    \item Der Code muss mit den folgenden zusätzlichen Flags compiliert werden:\\-fprofile-arcs -ftest-coverage
    \item Anschliessend kann gcov ein Object- oder Programmfile als Parameter übergeben werden
\end{itemize}
\textbf{Beispiel:}
\begin{lstlisting}[style=bash]
$ gcc -fprofile-arcs -ftest-coverage tmp.c
$ ./a.out
$ gcov tmp.c
90.00% of 10 source lines executed in file tmp.c
Creating tmp.c.gcov.
$
\end{lstlisting}
Output tmp.c.gcov:
\begin{lstlisting}[style=bash]
    -: 0:Source:tmp.c
    -: 0:Graph:tmp.gcno
    -: 0:Data:tmp.gcda
    -: 0:Runs:1
    -: 0:Programs:1
    -: 1:#include <stdio.h>
    -: 2:
    -: 3:int main (void)
    1: 4:{
    1: 5: int i, total;
    -: 6:
    1: 7: total = 0;
    -: 8:
   11: 9: for (i = 0; i < 10; i++)
   10: 10: total += i;
    -: 11:
    1: 12: if (total != 45)
#####: 13: printf ("Failure\n");
-: 14: else
1: 15: printf ("Success\n");
1: 16: return 0;
-: 17:}
\end{lstlisting}

\subsection{Performance Profiling - Valgrind}
Valgrind ist eine Tool Suite, die einige Debugging- und Profiling-Tools zur Verfügung stellt.

\subsubsection{Memcheck}
Memcheck detektiert Speichermanagementprobleme
\begin{itemize}
    \item Alle Schreibe- und Lesezugriffe auf den Speicher werden überprüft
    \item malloc/new/free/delete wird überprüft, um Speicherlecks zu detektieren
    \item Arraygrenzen-Überwachung von Memcheck ist für den Heap-Speicher ausgelegt und nicht für Arrays auf dem Stack oder für statische Arrays
    \item \url{http://valgrind.org/docs/manual/mc-manual.html}
\end{itemize}
\subsubsection{Callgrind}
Callgrind kann die Laufzeit eines Programms analysieren
\begin{itemize}
    \item In der Standardkonfiguration wird die Anzahl der ausgeführten Instruktionen aufgezeichnet
    \item Die aufgezeichnete Anzahl der Instruktionen wird zudem mit Sourcecode-Zeilen und Funktionsaufrufen in Beziehung gesetzt
    \item Cachesimulation (Cachegrind) kann aktiviert werden, die weitere Informationen über die Laufzeitperformance des Programms geben kann
    \item \url{http://valgrind.org/docs/manual/cl-manual.html}
\end{itemize}
\subsubsection{Valgrind-Output}
\begin{itemize}
    \item Das Ergebnis der Laufzeitanalyse wird in eine Datei geschrieben
    \item Die Datei kann dann mit dem Programm KCachegrind visualisiert werden
\end{itemize}
