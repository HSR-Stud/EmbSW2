\section{Simulation}

\subsection{Motivation}
\begin{itemize}
	\item Das Verhalten kann mittels Simulationen analysiert werden
	\item Mit Simulationen können Zeiten verkürzt werden
	\item Durch Simulationen kann die Reaktion der Steuerung auf Ausnahmesituationen überprüft werden
\end{itemize}

\subsection{Eingabewerte}
\subsubsection{Vernünftige Eingabewerte}
\begin{itemize}
	\item Reale, Eingabewerte entsprechen möglichst gut der Realität
	\item Häufig sollen bei Simulationen zufällige Eingabewerte verwendet werden, die nicht notwendigerweise gleichverteilt sind.
	\begin{itemize}
		\item Gleichverteilung
		\item Normalverteilung (Gauss'sche Glockenkurve)
		\item Rayleigh-Verteilung
		\item Weibull-Verteilung
	\end{itemize}
	\item Die Simulationen sollen reproduzierbar sein.
\end{itemize}

\subsection{Pseudozufällige Zahlen}
\begin{itemize}
	\item Echt zufällige Zahlen können nur durch einen physikalischen Prozess erzeugt werden, z.B. durch Würfel oder Thermisches Rauschen
	\item Eine deterministische Maschine wie der Computer kann keine echt zufälligen Zahlen generieren.
	\item Pseudozufällige Zahlen sind deterministisch. Bei einem gegebenen	Anfangswert (random seed) und gegebener Funktion ist die zufällige Zahlenfolge immer identisch.
\end{itemize}

\subsubsection{Pseudozufallszahlen-Generator}

\textbf{Forderungen}
\begin{itemize}
	\item Gleichverteilung
	\item nächste Zahl nicht vorhersehbar
\end{itemize}
\begin{equation}
	r_{i+1} = (a \cdot r_i + c)mod (m)
\end{equation}

\subsubsection{Mapping}
rand() = Gleichverteilung\\

\begin{tabular}{|l|c|}
	\hline
	\textbf{rand()}                              & liefert ganzzahlige Werte im Bereich [0, RAND\_MAX] \\ \hline
	\textbf{(double)rand()/RAND\_MAX}            &         liefert Werte im Bereich [0, 1.0]          \\ \hline
	\textbf{(double)rand()/RAND\_MAX $\cdot$(b-a) + a} &          liefert Werte im Bereich [a, b]           \\ \hline
\end{tabular} \\\\

\textbf{Bei einer Umwandlung in eine ganze Zahl muss beachtet werden, ob runden oder abschneiden die richtige Operation ist.}\\

$static\_cast<int> (static\_cast<double> (rand()) / (RAND\_MAX + 1.0) \cdot (high - low + 1)) + low$


\subsection{Wahrscheinlichkeitsverteilungen}

\subsubsection{Normalverteilung}
\textbf{Zentrale Grenzwertsatz}\\
Die Summe von n unabhängigen, identisch verteilten Zufallszahlen stellt für n gegen unendlich eine Normalverteilung dar.

Eine Normalverteilung kann somit angenähert werden, wenn eine grosse Anzahl gleichverteilter Zahlen summiert wird.

\begin{lstlisting}[style=C]
	do
	{
		v1 = 2 * static_cast<double> (rand()) / RAND_MAX - 1.0;
		v2 = 2 * static_cast<double> (rand()) / RAND_MAX - 1.0;
		s = v1 * v1 + v2 * v2;
	} while (s >= 1.0 || s == 0.0);
	
	double temp = sqrt(-2.0 * log(s) / s);
	return lround(v1 * temp * sdev + mean);
\end{lstlisting} 

\subsubsection{Rayleigh Verteilung}
Eine Rayleigh-verteilte Zahl R kann aus zwei normalverteilten Zahlen $N_1$, $N_2$ mit Mittelwert 0 und Standardabweichung $\sigma$ wie folgt erhalten werden:

\begin{equation}
	R(\sigma) = \sqrt{N_1^2 + N_2^2}
\end{equation}

 \begin{equation}
 	R(\sigma) = \sigma \sqrt{-2 \ln(U)} \qquad U \neq 0
 \end{equation}\\

\begin{lstlisting}[style=C]
	do
	{
		v1 = 2 * static_cast<double> (rand()) / RAND_MAX - 1.0;
		v2 = 2 * static_cast<double> (rand()) / RAND_MAX - 1.0;
		s = v1 * v1 + v2 * v2;
	} while (s >= 1.0 || s == 0.0);
	
	double temp = sqrt(-2.0 * log(s) / s);
	w1 = (v1 * temp * sdev);
	w2 = (v2 * temp * sdev);
	return lround(sqrt(w1 * w1 + w2 * w2));

\end{lstlisting} 

\subsubsection{Weibull Verteilung}
Die Dichtefunktion der Weibull-Verteilung kann mit 2 Parametern $\alpha$, $\beta$ wie folgt beschrieben werden:

\begin{equation}
	W_{\alpha,\beta}(x) = \alpha \cdot \beta \cdot x^\beta \cdot e^{-\alpha \cdot x^\beta} 
\end{equation}

\begin{itemize}
	\item $\beta = 1$:  Exponentialverteilung
	\item $\beta = 3.4$:  Normalverteilung
	\item $\beta > 1$:  Alterndes System
\end{itemize}


Eine Weibull-verteilte Zahl $W$ kann aus einer in $[0, 1.0[$ gleichverteilten Zahl $U$ mit folgender Formel berechnet werden:

\begin{equation}
	W(\alpha,\beta) = \beta \cdot (-\ln(1.0-U))^\frac{1}{\alpha}  \qquad U \neq 1.0
\end{equation}\\


\textbf{Anwendung der Weibull-Verteilung}

\begin{itemize}
	\item Untersuchung von Lebensdauern in der Qualitätssicherung
	\item Materialermüdungen von spröden Werkstoffen
	\item Ausfälle von elektronischen Bauteilen
	\item Verteilung von Windgeschwindigkeiten (oft wird hierzu auch die Rayleigh-Verteilung herangezogen)
	\item Allgemein: Systeme mit im Laufe der Zeit steigender Ausfallrate (Systeme mit konstanter Ausfallrate sind exponentialverteilt)
\end{itemize}
