\section{Simulation}

\subsection{Motivation}
\begin{itemize}
	\item Das Verhalten kann mittels Simulationen analysiert werden
	\item Mit Simulationen können Zeiten verkürzt werden
	\item Durch Simulationen kann die Reaktion der Steuerung auf Ausnahmesituationen überprüft werden
\end{itemize}

\subsection{Eingabewerte}
\subsubsection{Vernünftige Eingabewerte}
\begin{itemize}
	\item Reale, Eingabewerte entsprechen möglichst gut der Realität
	\item Häufig sollen bei Simulationen zufällige Eingabewerte verwendet werden, die nicht notwendigerweise gleichverteilt sind.
	\begin{itemize}
		\item Gleichverteilung
		\item Normalverteilung (Gauss'sche Glockenkurve)
		\item Rayleigh-Verteilung
		\item Weibull-Verteilung
	\end{itemize}
	\item Die Simulationen sollen reproduzierbar sein.
\end{itemize}

\subsection{Pseudozufällige Zahlen}
\begin{itemize}
	\item Echt zufällige Zahlen können nur durch einen physikalischen Prozess erzeugt werden, z.B. durch Würfel oder Thermisches Rauschen
	\item Eine deterministische Maschine wie der Computer kann keine echt zufälligen Zahlen generieren.
	\item Pseudozufällige Zahlen sind deterministisch. Bei einem gegebenen	Anfangswert (random seed) und gegebener Funktion ist die zufällige Zahlenfolge immer identisch.
\end{itemize}

\subsubsection{Pseudozufallszahlen-Generator}

\textbf{Forderungen}
\begin{itemize}
	\item Gleichverteilung
	\item nächste Zahl nicht vorhersehbar
\end{itemize}
\begin{equation}
	r_{i+1} = (a \cdot r_i + c)mod (m)
\end{equation}

\subsubsection{Mapping}
rand() = Gleichverteilung\\

\begin{tabular}{|l|c|}
	\hline
	\textbf{rand()}                              & liefert ganzzahlige Werte im Bereich [0, RAND\_MAX] \\ \hline
	\textbf{(double)rand()/RAND\_MAX}            &         liefert Werte im Bereich [0, 1.0]          \\ \hline
	\textbf{(double)rand()/RAND\_MAX $\cdot$(b-a) + a} &          liefert Werte im Bereich [a, b]           \\ \hline
\end{tabular} \\\\

\textbf{Bei einer Umwandlung in eine ganze Zahlmuss beachtet werden, ob runden oder abschneiden die richtige Operation ist.}\\

$static\_cast<int> (static\_cast<double> (rand()) / (RAND\_MAX + 1.0) \cdot (high - low + 1)) + low$


\subsection{Wahrscheinlichkeitsverteilungen}
