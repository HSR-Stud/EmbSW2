%------------------------------------------------------------------------------------------------
%Simulationen
%------------------------------------------------------------------------------------------------
\section{Simulationen}
\subsection{Motivation}
\begin{itemize}
\item Viele Prozesse sind zu komplex, um analytische L"osung f"ur ihr Verhalten zu finden
	\begin{itemize}
	\item Verhalten kann mittels Simulation analysiert werden
	\end{itemize}
\item Bei langsamen Prozessen dauert das Testen viel zu lange
	\begin{itemize}
	\item Simulationen verk"urzen diese Zeit
	\end{itemize}
\item Herstellung von Ausnahmesituation mit realem System sehr schwierig
	\begin{itemize}
	\item Durch Simulation kann die Reaktion auf solche Situationen "uberpr"uft werden
	\end{itemize}
\end{itemize}

\subsection{Eingabewerte}
Simulator erh"alt Eingabewerte worauf seine Reaktion darauf gepr"uft wird. 
\subsubsection{Reale Eingabewerte}
H"aufig werden zuf"allige Eingabewerte gew"ahlt, die nicht unbedingt gleichverteilt sind.\\
H"aufige Verteilungen:\\
\begin{itemize}
\item Gleichverteilung
\item Normalverteilung (Gauss)
\item Rayleigh-Verteilung
\item Weibull-Verteilung
\end{itemize}
Die Wahrscheinlichkeitsverteilung gibt an, wie h"aufig ein Wert einer Zufallsvariabel bei einem stochastischen Prozess vorkommt. 

\subsection{Pseudo-Random Numbers}
Echt zuf"allige Zahlen k"onnen nur durch einen physikalischen Prozess erzeugt werden (bspw. Thermisches Rauschen). Der Computer als deterministische Maschine kann nur pseudozuf"allige Zahlen generieren. 
Pseudozuf"allige Zahlen sind deterministisch, d.h. bei gegebenem Anfangswert (random seed) und gegebener Funktion ist die zuf"allige Zahlenfolge immer identisch.\\
Vorteil: Simulation ist reproduzierbar
Nachteil: Keine wirklche Zuf"alligkeit

\subsection{Pseudozufallszahlen-Generator}
Liefern gleichverteilte Werte in einem bestimmten Bereich. 
\begin{lstlisting}[style=C]
int rand(void); //Liefert gleichverteilte Zahlen im Bereich [0,RAND_MAX]
void srand(unsigned int seed) //Setzt Anfangswert (random seed)
\end{lstlisting}
\subsection{Mapping von Zufallszahlen auf beliebigen Bereich}
Zufallszahlen auf bestimmten Bereich mappen: 
rand(n) \% n $\rightarrow$ Bereich [0,n-1]

