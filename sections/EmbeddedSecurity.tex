%!TEX root = ../EmbSW2.tex
\section{Embedded Security}

\subsection{Goals and Non-Goals}
\begin{multicols}{2}
  Goals:
  \begin{itemize}
    \item Developing Secure Embedded Systems
    \item Designing Security into Embedded Systems
    \item Knowing the Security Fundamentals
  \end{itemize}
  \vfill\null
  \columnbreak
  Non-Goals:
  \begin{itemize}
    \item Designing your own security algorithms (NEVER try this!)
    \item Becoming a security specialist
    \item Learning and/or analyse cryptographic algorithms
  \end{itemize}
\end{multicols}

\subsection{What is security?}
Simply defined: A product's ability to protect data, assets and services from unauthorized access, while simultaneously making the product's assets and services available to authorized users.\\

Security: more than just behaving as ''expected''. It also means not doing unexpected or unsafe things in the face of an attack or malfunction.

\subsection{General Security Requirements}
\begin{description}
  \item[Confidentiality] (Vertraulichkeit)\\ Message can be understood only by the intended entities
  \item[Integrity] (Integrität)\\ Message is not altered/tampered by a third party
  \item[Authentication] (Authentisierung)\\ Ensures that the entities involved in an operation are who they claim to be
  \item[Non-Repudiation] (Nichtzurückweisbarkeit)\\ An entity cannot deny an action that it has performed
\end{description}

Remark: not all of these requirements are mandatory for all applications

\subsubsection{Embedded Systems are different}
PCs: may be more secure than ''typical'' embedded systems!
\begin{itemize}
  \item Users are tpyically weakest link
  \begin{itemize}
    \item Phishing, Social Engineering
    \item Bad habits (no VPN, passwords taped to monitor)
    \item Weak passwords
    \item Ignoring browser's certificate warnings
  \end{itemize}
\end{itemize}
Embedded systems: often no direct user interaction
\begin{itemize}
  \item Weakest link is hardware and/or firmware
  \begin{itemize}
    \item Last line of defense!
  \end{itemize}
  \item Our responsibility: advance product security!
\end{itemize}

\subsection{Hardware Support/Accelerators}

\subsubsection{Introducing the STM32F437IG}
Processor features:
\begin{itemize}
  \item Powerful core (Cortex M4)
  \begin{itemize}
    \item Memory Protection Unit (MPU)
  \end{itemize}
  \item Crypto Processor
  \begin{itemize}
    \item Performance, Power, Correctness
  \end{itemize}
  \item Analog True Random Number Generation
  \item Secure JTAG \& Internal Memories
  \item Secure Boot \& Secure Firmware Upload
  \item Network capable (100 Mbit Ethernet)
  \item Device-specific unique 96 bit identifier
  \begin{itemize}
    \item Can never be modified
    \item Guaranteed unique serial \# (not random)
    \item Used e.g. for device-specific key generation
  \end{itemize}
\end{itemize}

\subsubsection{ARM Cortex M4}
\begin{itemize}
  \item Powerful CPU running at 180MHz
  \begin{itemize}
    \item Cryptography, particularly public key, is CPU-intensive
    \item On-chip FPU
    \begin{itemize}
      \item Nice, but unused for crypto
    \end{itemize}
  \end{itemize}
  \item 2 privilege levels \& 2 separate stacks
  \begin{itemize}
    \item Separation of kernel/driver code from user code
  \end{itemize}
  \item Extensive fault-handling exceptions
  \begin{itemize}
    \item Bus Fault, Usage Fault, MPU Fault, Hard Fault, NMI
  \end{itemize}
  \item Widely-supported architecture
  \begin{itemize}
    \item Tools, RTOSs, 3rd-party libraries, etc.
  \end{itemize}
\end{itemize}

\subsubsection{ARM Cortex Memory Protection Unig (MPU)}
\begin{itemize}
  \item Partitions memory map into regions with attributes
  \item Traps invalid/disallowed memory accesses
  \begin{itemize}
    \item Programming Errors
    \begin{itemize}
      \item Write to Flash
      \item Write to holes in memory map
      \item Reads from write-only registers
    \end{itemize}
    \item Access Violations
    \begin{itemize}
      \item Execute Code from RAM
      \item Read/Write from Private Area
      \item Manipulation of Device/Peripheral
    \end{itemize}
  \end{itemize}
  \item Check your $\mu$C's errata sheet!
\end{itemize}

\subsubsection{Read Protection Modes}
Flash controller option bytes
\begin{itemize}
  \item Configure device's Read Protection Mode
  \begin{itemize}
    \item Level 0 - no protection
    \item Level 1 - read protection enabled
    \begin{itemize}
      \item Flash and Backup SRAM memories read-protected
      \item Limited debug ability
    \end{itemize}
    \item Level 2 - chip protection enabled
    \begin{itemize}
      \item Permanent \& irreversible
      \item JTAG interface disabled (fuses blown)
      \item Memories only accessible from code programmed in flash
      \item Boot from System ROM and from RAM: disabledTraps invalid/disallowed memory accesses
    \end{itemize}
  \end{itemize}
\end{itemize}

\subsubsection{Secure Firmware Update}
\begin{itemize}
  \item Firmware updates are common attack vectors
  \item Secure firmware update requirements
  \begin{itemize}
    \item Secrecy {\textemdash} object code transmitted in encrypted form ($\rightarrow$ maybe required)
    \item Authentication (integrity) ($\rightarrow$ mandatory)
    \begin{itemize}
      \item Code has not been tampered with or corrupted
      \item Code is from a ''live'' party that we trust
    \end{itemize}
  \end{itemize}
  \item 2 phases to secure firmware update
  \begin{itemize}
    \item Before shipping: Device is ''personalized''
    \item In field, device receives, decrypts, authenticates
  \end{itemize}
\end{itemize}

\subsubsection{STM32F437IG Crypto/Hash Processor (Hardware Acceleration)}
Built in symmetric ciphers
\begin{itemize}
  \item AES (Advanced Encryption Standard)
  \begin{itemize}
    \item CBC (cipher block chaining), GCM (galois counter), CCM (Counter with CBC-MAC) and CTR (counter) modes (Dont' use ECB (electronic codebook))
    \item AES-128, AES-192, AES-256
  \end{itemize}
  \item 3DES (Triple-DES, Data Encryption Standard)
  \begin{itemize}
    \item CBC (cipher block chaining) mode (Don't use ECB (electronic codebook))
    \item 64, 128, 192 bit keys
  \end{itemize}
  \item Cryptographic Hash Processor
  \begin{itemize}
    \item MD5, SHA-1, SHA-2
  \end{itemize}
  \item Message authentication
  \begin{itemize}
    \item HMAC
  \end{itemize}
\end{itemize}

\subsection{The security threats in embedded systems}
\begin{itemize}
  \item The attack surfaces are different
  \item The attacks are different
  \item The assets are different
\end{itemize}

Each product requires its own unique analysis

Different products, different concerns
\begin{itemize}
  \item Tampering, counterfeiting, reverse engineering, side channel attacks
  \item Loss of service, degraded performance
  \item Private data exposed, access granted
\end{itemize}

The solutions and countermeasures are different:
Big-iron approaches don't scale down very well
\begin{itemize}
  \item There is no standard set of ''must have'' defenses
  \begin{itemize}
    \item Depends on threats, assets, security goals
  \end{itemize}
\end{itemize}
\textbf{Can't} make any of these assumptions:
\begin{itemize}
  \item Updates can be periodically downloaded \& installed
  \item Memory and non-volatile storage (HDD) are plentiful
  \item CPU won't even break a sweat on public key crypto
  \item Safe physical environment
  \item Friendly user
\end{itemize}

\subsubsection{Example: Unreliable Random Number Generation}
Random number generators are used by many cryptographic and security primitives
\begin{itemize}
  \item Key generation, challenge-response, etc.
\end{itemize}
Unreliable RNGs are an opportunity for attacker
\begin{itemize}
  \item Keys can be predicted or discovered
  \item Challenges repeat over time
  \item Shuffling becomes predictable
\end{itemize}
1998 Aritona lottery (''Pick 3'')
\begin{itemize}
  \item No winning number ever had a ''9'' digit
\end{itemize}

\subsubsection{Example: Reliable but insecure}
Keypad for 4-digit PIN entry\\
Screen and buzzer\\
User's PIN is ''1234''\\
Keypad proven to never authenticate an incorrect PIN\\

Problem: The moment an incorrect digit is entered, buzzer sounds, PIN entry is re-started\\
\textbf{10000 tries just went down to 40}
