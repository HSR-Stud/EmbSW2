\section{Advanced C++}

\subsection{Static vs. Dynamic Binding}
\begin{itemize}
	\item Static binding (early binding, statische Bindung)
		\begin{itemize}
			\item bereits zur Compilezeit wird festgelegt, welcher (Elementfunktions-) Code
				ausgeführt wird (Normalfall)
			\item ''normale`` Klassen, templates
		\end{itemize}
	\item Dynamic binding (late binding, dynamische Bindung)
		\begin{itemize}
			\item erst zur Laufzeit wird in Abhängigkeit des Objekts festgelegt, welcher
				(Elementfunktions-) Code ausgeführt wird
			\item Run-Time Polymorphismus, virtual
			\item Laufzeit mehraufwand, weniger effizient
		\end{itemize}
\end{itemize}

\subsubsection{Virtuelle Memberfunktionen}

Soll die Funktion überschrieben werden können, muss \emph{virtual} verwendet werden.
Wird virtual nicht verwendet, wird die Funktion nur ''versteckt``.
\lstinputlisting{code/virttest.cpp}


\subsection[RAII]{RAII - RESOURCE ACQUISITION IS INITIALISATION}

\paragraph{Motivation}
Resourcen müssen vor Gebrauch angefordert und danach freigegeben werden.
Dazwischen könnte aber z.B. eine Exception auftreten.

\paragraph{Lösung}
Damit die Resource trotzdem freigegeben wird, kann sie mit einer Klasse gekappselt werden:
\begin{itemize}
	\item Konstruktor fordert die Resource an
	\item Destruktor gibt sie wieder frei
\end{itemize}


\paragraph{Heap-Objekte}
Für Heap-Objekte stellt die Boost-Library Templates für diesen Zweck zur Verfügung.

\begin{lstlisting}
void f()
{
	Person* p = new Person("irgendwer");
	// mach etwas mit p
	// was ist, wenn vorher eine Exception geworfen wird?
	delete p;
}
void f()
{
	boost::shared_ptr<Person> p(new Person("irgendwer"));
	// mach etwas mit p
	// Beim Verlassen des Blocks raeumt Destruktor von shared_ptr
	// automatisch auf und loescht die Person
}
\end{lstlisting}


\paragraph{Semaphoren} Für Semaphoren könnte die folgende Implementierung verwendet werden


\begin{lstlisting}
class Semaphore
{
public:
	Semaphore(int s=0): id(s) {getSem(s);}
	~Semaphore() {releaseSem(id);}
private:
	int id;
};
void f()
{
	// kritischer Abschnitt
	{
		Semaphore myS;
	} // hier wird Semaphore freigegeben
}
\end{lstlisting}


\subsection[pImpl]{pImpl - Pointer to Implementation}
\label{sec:pimpl}

\paragraph{Problem} Wie kann man die Implementierung einer Klasse so verstecken, dass man sie
ändern kann, ohne alle Module, welche die Klasse nutzen, bei einer Änderung
neu übersetzen zu müssen?
Nützlich zum Beispiel für eine shared library / DLL.

\paragraph{Lösung}
\begin{itemize}
\item Implementierung in separates File.
\item Im Header nur noch öffentliche Schnittstelle und Pointer zu Implementierung
\end{itemize}

\begin{lstlisting}
#ifndef HIDDENCOUNTER_H_ // public file
#define HIDDENCOUNTER_H_
#include <boost/shared_ptr.hpp>
class HiddenCounter
{
public:
	HiddenCounter(int i=0);
	void inc();
	int count() const;
	void reset();
private:
	boost::shared_ptr<class CounterImpl> pImpl;
};
#endif // HIDDENCOUNTER_H_

#include "HiddenCounter.h" // private file
class CounterImpl
{
public:
	CounterImpl(int i): counter(i) {}
	void inc() { ++counter; }
	int count() const { return counter;}
	void reset() { counter=0; }
private:
	int counter;
};
HiddenCounter::HiddenCounter(int i)
:pImpl(new CounterImpl(i)) {}
void HiddenCounter::inc() {pImpl->inc();}
int HiddenCounter::count() const
	{ return pImpl->count(); }
void HiddenCounter::reset() {pImpl->reset();}
\end{lstlisting}


\subsection{Performance}

\subsubsection{Variablen}
\begin{itemize}
	\item Wenn möglich immer Standardtypen verwenden (int, unsigned int), da
		diese den Registerbreiten entsprechen
	\item Typen wie short und char sparen kaum Platz und sind meistens
		langsamer, da sie speziell behandelt werden müssen.
	\item Werden definierte Bitlängen benötigt, sollten die Typen aus
		$<inttypes.h>$ bzw. $<stdint.h>$ (ab C99): 
		intN\_t,  uintN\_t z.b. uint8\_t \\
		Sie können auch selbst getypedeft werden.
\end{itemize}

Lokale Variablen sollte man so verwenden, dass sie in Registern gebraucht werden können.
Z.b. verhindern Pointer auf lokale Variablen, dass diese in ein Register gelegt werden.

\subsubsection{Lächerliches}
++i soll angeblich schneller sein als i++, da aber i++ viel hübscher aussieht sollte man es eher verwenden ;)

\subsubsection{C++ vs. C}
\begin{itemize}
	\item No-Cost C++ features
		\begin{itemize}
			\item Klassen
			\item Namespaces
			\item Statische Funktionen und Daten
			\item Nicht virtuelle Funktionen
			\item Überladung
			\item Default Parameter, es ist aber zu beachten, dass diese immer übergeben werden, bei grossen Objekten ist dies ineffizient:
				\begin{lstlisting}
void doThat(const std\dotsstring& name = "Unnamed"); // Bad
const std\dotsstring defaultName = "Unnamed";
void doThat(const std\dotsstring& name = defaultName); // Better
				\end{lstlisting}
			\item  Konstruktoren / Destruktoren, sofern korrekt verwendet
			\item Einfache Vererbung
		\end{itemize}
	\item Low-Cost C++ features
		\begin{itemize}
			\item Exceptions, jedoch kann anderes Error-Handling, z.B. via return values auch teuer sein.
		\end{itemize}
	\item High-Cost C++ features, wenn falsch verwendet
		\begin{itemize}
			\item Temporäre Objekte, Objektkopieen (z.B. gibt a+b ein temporäres Objekt zurück), kann einfach vermieden werden
			\item Templates können zu Code-Bloat führen
		\end{itemize}
\end{itemize}

\subsubsection{Inlining}
\begin{itemize}
	\item Vorteile
		\begin{itemize}
			\item Kein Funktionsaufruf
			\item daher schneller
		\end{itemize}
	\item Nachteile
		\begin{itemize}
			\item Mehr Code
			\item schwerer zu debuggen
		\end{itemize}
	\item Beschränkungen
		\begin{itemize}
			\item Compiler kann inlining ignorieren
			\item Inlining unmöglich für:
				\begin{itemize}
					\item Funktionen auf die ein Pointer zeigt
					\item Virtuelle Funktionen
					\item Rekursive Funktionen
				\end{itemize}
			\item Linker kann selten Inlinen, daher inlinefunktionen im Header definieren
		\end{itemize}
\end{itemize}

\subsubsection{Polymorphismus}
\begin{itemize}
	\item Runtime
		\begin{itemize}
			\item Vererbung virtuelle Funktionen
			\item Teurer Laufzeitaufwand
		\end{itemize}
	\item Link-Time
		\begin{itemize}
			\item Separat complierte Funktionen mit gleicher definition (z.B.
				Hardware abhängiger code)
			\item Statisches oder dynamisches linken, siehe \nameref{sec:pimpl}
		\end{itemize}
	\item Compile-Time
		\begin{itemize}
			\item Templates, \#ifdef's, const expressions
		\end{itemize}
\end{itemize}

\subsubsection{Dynamic Memory Managment}
Die vier Sorgen:
\begin{itemize}
\item Schnelligkeit: Sind \textbf{new/delete/malloc/free} schnell genug?
\item Fragmentation: Entstehen im Heap unbrauchbar kleine Bereiche?
\item Speicherlecks: Werden einige Bereiche nicht frei gegeben?
\item Speicherersch"opfung: Wird eine Allokation nicht gemacht?
\end{itemize}

Allocation Strategies: 
\begin{itemize}
\item Fully Static Allocation
\item LIFO Allocation
\item Pool Allocation
\item Block Allocation
\item Region Allocation
\end{itemize}

Fully Static Allocation: 
\begin{itemize}
\item Daten nicht auf dem Heap $\rightarrow$ Exakte bzw. max. Anzahl der Obejkte ist determinierbar. 
\item \textbf{Speed}: praktisch unendlich, deterministisch
\item \textbf{Fragmentation}: Nicht m"oglich
\item \textbf{Memory leaks}: Nicht m"oglich
\item \textbf{Memory exhaustion}: Nicht m"oglich
\end{itemize}

LIFO Heap Allocation: 
\begin{itemize}
\item Dynamische Allokation ausserhalb des runtime Stacks $\rightarrow$ Diese Allokationen sind immer LIFO (wie beim Stack)
\item \textbf{Speed}: Sehr schnell, deterministisch 
\item \textbf{Fragmentation}: M"oglich, einfach zu erkennen
\item \textbf{Memory leaks}: M"oglich, einfach zu erkennen
\item \textbf{Memory exhaustion}: M"oglich
\end{itemize}

Pool Allocation: 
\begin{itemize}
\item Heap Allokationen sind alle gleich gross (bzw. alle so gross, wie das gr"osste Element) 
\item \textbf{Speed}: Sehr schnell, deterministisch 
\item \textbf{Fragmentation}: Nicht m"oglich
\item \textbf{Memory leaks}: M"oglich
\item \textbf{Memory exhaustion}: M"oglich
\end{itemize}

Block Allocation: 
\begin{itemize}
\item Verschiedene Sets von Pools mit verschiedenen Gr"ossen 
\item \textbf{Speed}: Schnell, praktisch deterministisch 
\item \textbf{Fragmentation}: Nicht m"oglich
\item \textbf{Memory leaks}: M"oglich
\item \textbf{Memory exhaustion}: M"oglich
\end{itemize}

Region Allocation: N"utzlich wenn alle Heapobjekte simultan frei gegeben werden. 

\subsubsection{C++ and ROM}
Folgendes ist ROMable: 
\begin{itemize}
\item Wert ist vor Laufzeit bekannt
\item Wert kann zur Laufzeit nicht modifiziert werden
\item Warum sind \#define schlecht? $\rightarrow$ Sie respektieren den Scope nicht, sie sind weder private noch protected $\rightarrow$ Besser mit const. 
\item Werden geROMt, wenn: Als const definiert, wenn sie keine mutierenden Daten enthalten, Wenn sie mit bekannten Werten initialisiert werden w"ahrend der Compilation. 
\end{itemize}


\subsubsection{Modeling Memory-Mapped I/O}
Viele Systeme haben ihre I/O (\textit{ memory-mapped IO devices}) im Programm-Adresse Bereich. Mit C++ ist es einfach diese \textit{ memory-mapped IO devices} mithilfe von Objekten anzusteuern.\\ \\

\textbf{Modeling Memory-Mapped IO}
\begin{itemize}
	\item Atomic reads/writes may require explicit synchronization.
	\item Individual bits may sometimes be read-only, other times write-only.
	\item Clearing a bit may require assigning a 1 to it.
	\item One status register may control more than one data register.
		E.g., bits 0-3 are for one data register, bits 4-7 for another.	
\end{itemize}

\textbf{Modeling a Control Register}

\begin{lstlisting}
enum { bit0 = 0x1, bit1 = 0x2, ... , bit31 = 0x80000000 };

class ControlReg {
public:
	bool ready() const 	{ return regValue & bit0; }
	bool interruptsEnabled() const { return regValue & bit2; }
	void enableInterrupts() { regValue |= bit2; }
	void disableInterrupts(){ regValue &= ~bit2; }
private:
	volatile unsigned regValue;		// data in the register may change
									// outside program control
};
\end{lstlisting}

All functions are inline, so their existence should incur no cost.\\ \\

\textbf{Placement new}

Operator new’s fundamental job is not to allocate memory, it’s to identify where an object should go.

\begin{itemize}
	\item You can pass operator new where you want to put something, and it will return that location
	\item This form of operator new is called placement new.
			It’s a standard form available everywhere.
	
\end{itemize}

Beispiel mit Pointer:
\begin{lstlisting}
ControlReg * const pcr = new (reinterpret_cast<void*>(0xFFFF0000)) ControlReg;

while (!pcr->ready()) ; 			// wait until the ready bit is on
pcr->enableInterrupts(); 			// enable device interrupts
if (pcr->interruptsEnabled())... 	// if interrupts are enabled...

\end{lstlisting}

Beispiel mit Referenz:
\begin{lstlisting}
ControlReg& cr = *new (reinterpret_cast<void*>(0xFFFF0000)) ControlReg;

while (!cr.ready()) ; 				// wait until the ready bit is on
cr.enableInterrupts(); 				// enable device interrupts
if (cr.interruptsEnabled()) ... 	// if interrupts are enabled...
\end{lstlisting}
\hspace*{\fill} \\\\
Mit \textit{placement new} wird kein Speicher alloziert.\\ \\

\textbf{Raw Casts}
\begin{lstlisting}
ControlReg *pcr = reinterpret_cast<ControlReg*>(0xFFFF0000);	// pointer version

ControlReg& cr = *reinterpret_cast<ControlReg*>(0xFFFF0000);	// reference version
\end{lstlisting}


\textbf{Placement new vs. Raw Casts}


\textit{Placement new} wird normalerweise dem \textit{ raw reinterpret_cast} Vorgezogen.








